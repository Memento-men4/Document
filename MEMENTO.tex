\documentclass[conference]{IEEEtran}
\IEEEoverridecommandlockouts
\usepackage{cite}
\usepackage{amsmath,amssymb,amsfonts}
\usepackage{algorithmic}
\usepackage{graphicx}
\usepackage{kotex}
\usepackage{textcomp}
\usepackage{color}
\usepackage{xcolor}
\usepackage{graphicx}
\usepackage{caption}
\usepackage{tabularx}
\hbadness=99999 
\vbadness=99999 
\hfuzz=20pt
\def\BibTeX{{\rm B\kern-.05em{\sc i\kern-.025em b}\kern-.08em
    T\kern-.1667em\lower.7ex\hbox{E}\kern-.125emX}}
\begin{document}

\title{MEMENTO \\
- The Technical Solution for 'Youngzheimer'\\
}

\author{\IEEEauthorblockN{SUK CHEOL LEE}
\IEEEauthorblockA{\textit{College of Engineering} \\
\textit{Hanyang University}\\
\textit{Dept.of Information Systems}\\
Seoul, Korea \\
tjrcjf9@hanyang.ac.kr}
\and
\IEEEauthorblockN{HA NUEL LEE}
\IEEEauthorblockA{\textit{College of Engineering} \\
\textit{Hanyang University}\\
\textit{Dept.of Information Systems}\\
Seoul, Korea \\
lsk020@hanyang.ac.kr}
\and
\IEEEauthorblockN{SE HEE JEONG}
\IEEEauthorblockA{\textit{College of Engineering} \\
\textit{Hanyang University}\\
\textit{Dept.of Information Systems}\\
Seoul, Korea \\
sjsk04230000@hanyang.ac.kr}
\and
\IEEEauthorblockN{JUN SANG CHO}
\IEEEauthorblockA{\textit{College of Engineering} \\
\textit{Hanyang University}\\
\textit{Dept.of Information Systems}\\
Seoul, Korea \\
endlesa@hanyang.ac.kr}
}

\maketitle

\begin{abstract}
Our team is trying to develop MEMENTO, which is an user-friendly application that will help prevent Youngzheimer (Young + Alzheimer). Alzheimer, which is one of a type in dementia, has grown its problem on a rapid speed. The typical early symptom of dementia is memory loss, which gradually forgets what you have experienced and decreases your judgment as time goes by. Among them, Alzheimer's-type dementia accounts for more than half of all dementia patients, and nerve cells in the brain slowly decline, disappearing brain tissue, causing brain atrophy and dementia symptoms. According to the World Health Organization (WHO), the global dementia population is about 50 million, and it is expected to more than triple by a number of 152 million in 2050. Since dementia has no perfect treatment, early diagnosis, management, and maintenance are considered as a key factor to prevent dementia. Recently, the number of people suffering from excessive forgetfulness, an early symptom of dementia, is increasing rapidly, especially in the younger generation. Dementia is usually known as a common disease that occurs frequently in the elderly over the age of 65, but recent research in Korea has shown a totally different aspect. Symptoms such as severe forgetfulness, which is an early sign of a dementia, occurred frequently among 20s and 30s in Korea, as a high proportion of 43.9\%. Dementia that occurs from people aged lower than 65 years old is called "Young Onset Dementia," which is much faster and more dangerous than dementia of the elderly, but there are very few ways to deal with it. As a result, "Youngzheimer," a combination of Alzheimer's' newly coined words "young" and "Alzheimer’s," was also born. According to the study, the younger generations such as M-Z generation has a very high overall ranking on their health care. There is a phrase such as early care syndrome, which is taking care of their health early due to global diseases such as COVID-19. Also, they are very open-minded in investing their time and money in taking care of their health, and long for ‘healthy pleasure’, which stand for happy health care. High-level of interest in health care among younger generations will certainly work well for MEMENTO, and it is expected to elicit a positive response. And since MEMENTO was chosen be designed as an application, which is a platform that possesses one of the most biggest potential to be distributed to people most quickly, effectively, and widely, out application is expected to obtain user's interest in a glance. Based on the data analyzed in the user's daily life, MEMENTO will be a user-friendly application, such as asking questions to prevent dementia in the younger generation and providing feedback and results accordingly. 
\end{abstract}

\begin{IEEEkeywords}
Youngzheimer, Dementia, Health Care, User-Friendly Application, Prevention, MEMENTO.
\end{IEEEkeywords}

\begin{table} [h]
    \caption{Task Distributions for Each Member - 1}
    \centering
    \begin{tabular}{l|l|l}
    \hline
    \textit{\textbf{Roles}} & \textit{\textbf{Name}} & \textit{\textbf{Description}}
    & & & \\ 
    \hline
   \textit{\textbf{\begin{tabular}[c]{@{}l@{}}Software \\ Developer\\(Front-End)\end{tabular}}} & \textit{\textbf{\begin{tabular}[c]{@{}l@{}}SUK \\ CHEOL \\ LEE\end{tabular}}}& \begin{tabular}[c]{@{}l@{}}A Software Developer(Front-End) \\ uses web languages such as CSS, \\ HTML, and JavaScript to create \\ websites and applications. Anything\\that a user sees and click is a work\\ done by Front-End developer. \\ Ensuring user to easily access and\\interact with the site or app is their \\ primary goal. This process is done by \\  combining technology and, design, \\ and programming skills that determines\\ the appearance of a  website, as well as\\ taking care of any errors. Creating the\\ user interface  (UI) that determines what \\each part  of a site or application does\\ and how  it will look is their primary\\ focus.  Developer (Front-End) determines \\ where to place images, what the \\ navigation should look like, and how \\ to present the site. Much of their work\\ involves ensuring the appearance and\\ layout of the site or application is\\ easy to navigate and intuitive for \\the user. This role requires creativity,\\ problem-solving ability, communication\\ skills between others, and smooth \\teamwork.\end{tabular} \\ \hline
   \textit{\textbf{\begin{tabular}[c]{@{}l@{}}Software\\ Developer\\(Back-End)\end{tabular}}} & \textit{\textbf{\begin{tabular}[c]{@{}l@{}}HA \\ NUEL \\ LEE\end{tabular}}}& \begin{tabular}[c]{@{}l@{}}A Software developer (Back-end) is \\responsible for writing the web services\\ and APIs used by front-end developers\\ and mobile app developers. This role\\ oversees the server-side web application\\ logic as well as the integration of the\\ front-end part. Besides being in charge of \\the server-side logic, their primary focus \\is to define and maintain the central\\ database, making sure that it has high\\ performance and responsiveness to \\requests from the front-end. This\\ role must understand the structure of\\ their central database and software and\\ focus on features or tasks to make\\ the development of the software possible.  \end{tabular} \\ \hline
   \end{tabular}
\end{table}

\clearpage
\begin{table} [h]
    \setlength{\tabcolsep}{1pt}.
    \caption{Task Distributions for Each Member - 2}
    \centering
    \begin{tabular}{l|l|l}
    \hline
    \textit{\textbf{Roles}} & \textit{\textbf{Name}} & \textit{\textbf{Description}}
    & & & \\ 
    \hline
   \textit{\textbf{\begin{tabular}[c]{@{}l@{}}Software\\ Developer\\(Machine\\Learning)\end{tabular}}} &\textit{\textbf{\begin{tabular}[c]{@{}l@{}}SE \\ HEE \\ JEONG\end{tabular}}} & \begin{tabular}[c]{@{}l@{}} A Software developer (Machine learning)\\ works with algorithms, data, and artificial\\ intelligence. This role must make a \\research, build, and design the artificial\\ intelligence software responsible for\\ machine learning. Maintaining and\\ improving artificial intelligence systems\\ is their primary focus. This role performs\\ data collection, cleaning, and preprocessing\\ to extract meaningful value, and utilize it\\ in training models and deploy them to\\ software. This role is responsible for\\ implementing machine learning algorithms\\ to a software adequately, must run AI\\ systems experiments and tests, design and\\ develop machine learning systems, and\\ perform statistical analysis.\end{tabular} \\ \hline
   \textit{\textbf{\begin{tabular}[c]{@{}l@{}}Project\\ Designer\\(Documentation)\end{tabular}}} & \textit{\textbf{\begin{tabular}[c]{@{}l@{}}JUN \\ SANG \\ CHO\end{tabular}}}& \begin{tabular}[c]{@{}l@{}} A product designer is responsible to design\\ a product with user-centered sight and must\\ have the ability to sympathize with and\\ understand the user's experiences. This plays\\ as a role in creating the overall framework\\ of a product or services. Product designer\\ is responsible for communicating smoothly\\ with other team members and create\\ collaboration. Product designer must focus\\ on the usability of a product or service and\\ modify the design of a product to obtain\\ better results. \end{tabular} \\ \hline
   \end{tabular}
\end{table}

\section{Introduction}
\subsection{Motivation}
The number of dementia patients worldwide is on the rise. According to a report by the World Health Organization (WHO), the current number of dementia patients is estimated to be about 50 million and will reach 152 million by 2050, which is more than three times of current number of patients. Dementia is commonly known as a disease that occurs only in the elderly over the age of 65, but in recent years, people in their 40s and 50s and as early as 30s have suffered from dementia, or early symptoms of dementia such as forgetfulness and frequent forgetting. Dementia that occurs from people aged lower than 65 years old is called "Young Onset Dementia," which is much faster and more dangerous than dementia of the elderly, but there are very few ways to deal with it. The word 'Youngzheimer', which combines young, which means youth, and Alzheimer, the most representative type of dementia, was also born at this moment. Currently, dementia-related apps are developed by the Central Dementia Center called '치매체크', but they are mainly for the elderly who are already suffering from dementia, not for preventing purpose. And even they are criticized for frequent errors and insufficient optimization. The incidence of dementia among young people is increasing. However, currently most of the supports and systems or infrastructures related to dementia are only for the elderly, and it can be said that there are almost no supports or systems for the young. Dementia, one of the world's top 10 causes of disease and death, has yet to be treated properly and has no other means than preventing it with regular and healthy lifestyles. According to a dementia recognition study conducted by the Central Dementia Center and Gallup Korea, one in three adults chose dementia among the most feared diseases, with 43\% in their 60s and older ranking first. Everyone wants to avoid the cruel reality that it damages not only themselves but also families who take care of themselves and forget their names and surnames. However, the incidence of dementia in Korea is about 10\% of the population aged 65 or older, which is by no means a small number. Currently, the number of dementia patients in Korea is about 850,000, which is very high for a single disease. To be help in this urgent matter, our team has decided to create an application that not only provides useful information about dementia, but also functions that can actually help prevent dementia by revitalizing brain activities. Because dementia can be a very sensitive issue for some people, our team is trying to provide functions such as quiz from analyzed data based on users' daily activities. Therefore, our team developed its own dementia prevention app, MEMENTO, to help young people create regular and healthy habits and raise awareness of the risk of dementia and related information. MEMENTO will serve as a tool to stop Youngzheimers with a user-friendly approach, and later contribute to public health and welfare.

\subsection{Problem Statement}\label{SCM}
\begin{itemize}
\item [1] There are many similar applications that helps prevent brain diseases such as ‘dementia check’ created by national institute of dementia and national medical center, center of dementia. However, there’s currently no application that focuses on younger generation’s Youngzheimer.\\
\item [2] We should make an approach carefully since dementia can be a very stimulating subject for the younger generation. Apart from high interest in health care, the views of the younger generation on the  dementia could be very negative and are highly likely to be clearly divided. Emphasizing the role of MEMENTO is important to attract users among younger generation.\\
\item [3] The absolute number of people affected by, dying, or remaining disabled from neurological disorders such as dementia over the past 25 years has been increasing globally.\\
\item [4] Small actions such as going over one’s daily schedule and playing small games that activates brain has been proven effective in increasing brain health and preventing diseases\\
\item [5] Distribution rate of smartphones in Korea have reached 97\% in 2022, so application is the most effective and convenient tool to help prevent Youngzheimer.\\
\end{itemize}

\subsection{Solution}\label{SCM}
\begin{itemize}
\item We will create a differentiation of our application through the daily timeline and the quizzes created based on daily activities of a user.\\
\item Our application focuses on the prevention of Youngzheimer, by alleviating it’s early symptoms such as forgetfulness. Our program will not be used as a tool to determine whether a user is a dementia patient, which has a high possibility of provoking user’s unpleasantness.\\
\item We will increase the inflow and interest of our application by providing users with information that helps them manage their health from an early stage. These information includes such as vitamin recommendation, exercise recommendation, improvement of eating habits, other good habits, and overall information about dementia, increasing the alertness of a user.\\
\end{itemize}

\subsection{Research on Related Software}\label{SCM}
\begin{itemize}
A. SILVIA\\
\item SILVIA is an application for the purpose of preventing and treating degenerative brain diseases that provides various functions to improve brain health. Silvia's key features include initial questioning, habit management, expert counseling, health information, exercise content, brain training games, customized routine habit formation, memo functionality, meditation and stretching recommendations, and weekly statistics. SILVIA is currently available in both Android and iOS platforms.\\
\\
B. NeuroNation\\
\item NeuroNation is an application that strengthens the user's brain through tests in various sectors. It improves memory, strengthens attention, helps logical thinking, and increases speed of thought through attention tests, memory tests, reasoning tests, symbolism tests, and math skills tests. NeuroNation is currently available in both Android and iOS platforms.\\
\\
C. 치매체크\\
\item 치매체크 is an application developed by collaboration of Ministry of Health and Welfare of Korea and the Central Dementia Center. It’s main functions include checking the risk of dementia, an encyclopedia of dementia, a function to find the elderly through GPS, a function to manage dementia people, a care diary function, facility information to help dementia patients, and messages containing hope. 치매체크 is currently available in both Android and iOS platforms.\\
\end{itemize}

\section{REQUIREMENT ANALYSIS}
\subsection{Voice Recording}
MEMENTO supports voice recording function, which can be recorded whenever a user touches the record button in the screen. Touching the screen immediately triggers GPS function and saves the location information and time information. These recorded file, location information, time information are later automatically sent, and saved into the database.\\
\subsection{Daily Quiz}
According to the timeline and location information, and voice recordings that user have made in function A(Voice Recording) will be converted into a simple quizzes. User will be able to make improve their memory by solving the questions and will be able to go over their daily activities in a more interesting manner.\\
\subsection{Timeline}
User will be able to access daily timeline by touching a certain day in Function F(Calendar). In this timeline, information that converts the user's voice recording stored from Function A (Voice Recording) into text and information that sets the time and day after registering the user's LG home appliance in our application is stored. \\
\subsection{Prevention Game}
MEMENTO will provide three types of games that will help prevent Youngzheimer by an area of improving memory, improving calculation ability, improving focus by games that are officially selected by the Central Dementia Center and the Ministry of Health and Welfare. It makes it easy to approach the very sensitive subject of dementia through games, a medium that is very familiar to the public, especially to younger generations. As a result, the inflow of the younger generation and the application usage frequency can be increased in a high volume.
\subsection{Self-Diagnosis}
MEMENTO will be providing Youngzheimer self-diagnosis questions, similar to the ones that are officially adopted by the Ministry of Health and Welfare of the Republic of Korea. Users of the application will be able to conduct tests on their memory periodically, and check their current status by the out coming results of the test.
\subsection{Calendar}
User will be able choose whatever date they want from the calendar. The calendar will be consisted of selectable year, and month. When a user clicks a date, the application will show timeline based on the location information, time information and recorded information made from Function A(Recording).
\subsection{Useful Information}
MEMENTO provides various information such as food and vitamins recommendation, or dementia prevention center that helps prevent Youngzheimer. By acknowledging the potential danger and seriousness of dementia, user will be able to stay motivated and create good habit to stay far away from dementia.

\section{DEVELOPMENT ENVIRONMENT}
\subsection{Choice of Software Development Platform}

Our team will develop application in the environment of Mac operating system and Windows operating system. To create hybrid application that works for both iOS and Android platform, we will use React-Native framework which is based on JavaScript. We will also use Spring framework, which is based on JAVA, because it provides high effectiveness in storing and loading data in building a real-time server. As MySQL provides a database framework, it will be more helpful for the Spring framework.

\begin{table}[h]
    \caption{Tools and Language Choice}
    \centering
    \begin{tabular}{c|l}
    \hline
    \multicolumn{1}{l|}{\textit{\textbf{Tools and Language}}} & \textit{\textbf{Reason}} 
     & & \\ \hline
    \textit{\textbf{React-Native}} & \begin{tabular}[c]{@{}l@{}}React Native is an open-source UI software\\ framework created by Meta Platform, Inc.\\ It is used to develop applications for\\ Android, Android TV, iOS, macOS, tvOS,\\ Web, Windows and UWP by enabling\\developers to use the React framework\\along with native platform capabilities. It is\\ alsobeing used to develop virtual reality\\ applications at Oculus. React Native has one\\ of the biggest feature of being able to create\\ native UI for Android and iOS using\\ JavaScript, and create high-quality UI faster\\ than using HTML. React Native communicates\\ with Native Thread over native bridges,\\ optimizing performance unlike web apps.\\ Using a method of communicating with\\ the native without using this web is called a\\ Hybrid App, and there are Xamarin, Native\\ Script, and flutter. \end{tabular} \\ \hline
    \textit{\textbf{Spring}} & \begin{tabular}[c]{@{}l@{}}Spring Framework is an open-source\\ application framework for the Java platform\\ and is a lightweight solution that provides\\comprehensive capabilities for developing\\ enterprise-class applications. Enterprise-class\\ development is a development aimed at the\\ enterprise if you put it your way. In other\\ words, an enterprise environment is a very\\ large environment in which large data\\ processing and transactions occur\\ simultaneously from multiple users. The\\ Spring Framework is a lightweight container\\ that stores and manages Java objects directly.\\ It manages the creation, destruction, and life\\ cycle of objects, and you can import and use\\ the required objects from the Spring container\\ at any time. This means that Spring is an IOC-\\based framework. \end{tabular} \\ \hline
    \textit{\textbf{Flask}} & \begin{tabular}[c]{@{}l@{}}Flask is one of the microweb frameworks built\\ on Python. Flask is a micro-web framework,\\ so it can be kept concise and expanded. Python\\ is often used to develop AI-related programs,\\ and Flask is useful when you want to process\\ certain images or images separately in Python\\ code while serving mainly with other language-\\based web frameworks such as Spring. When\\ you transfer a file from the Spring server to the\\ Flask server, you can receive the file from the\\ Flask server, process it as desired with Python\\ code, and return it to the Spring server.\end{tabular} \\ \hline
    \end{tabular}
    \renewcommand{\thetable}{\arabic{table}}
    \captionsetup{justification=centering}
\end{table}

\begin{table}[h]
    \centering
    \begin{tabular}{c|l}
    \hline
    \textit{\textbf{MySQL}} & \begin{tabular}[c]{@{}l@{}}MySQL is the most widely used open-source\\ database worldwide and is a database developed\\ and distributed by MySQL AB. Open-source relational\\ database management management systems (RDBMS)\\ using the standard database query language SQL \\(Structured Query Language), which are very fast,\\ flexible, and easy to use. It supports multiple users,\\ multiple threads, and provides an application interface\\ (API) for C, C++, Eiffel, Java, Pearl, PHP, Python\\ scripts, and more. It can be used on Unix, Linux,\\ and Windows operating systems. The Linux operating\\ system, apache server program, MySQL, and PHP\\ script language composition are free programs that are\\ well interworking and are developed open-source, so\\ they are widely used for general web development such\\ as homepages and Shopping Mall. \end{tabular} \\ \hline
    \end{tabular}
    \renewcommand{\thetable}{\arabic{table}}
    \captionsetup{justification=centering}
\end{table}
\break 

\begin{table}[h]
    \caption{Develop Environment}
    \centering
    \begin{tabular}{c|l}
    \hline
    \multicolumn{1}{l|}{\textit{\textbf{Name}}} & \textit{\textbf{Development environment}} 
     & & \\ \hline
    \textit{\textbf{SUK CHEOL LEE}} & \begin{tabular}[c]{@{}l@{}}MacOS Monterey 12.3.1\\React-Native\end{tabular} \\ \hline
    \textit{\textbf{HA NUEL LEE}} & \begin{tabular}[c]{@{}l@{}}MacOS Monterey 12.2.1\\Spring\\Flask\end{tabular} \\ \hline
    \textit{\textbf{SE HEE JEONG}} & \begin{tabular}[c]{@{}l@{}}MacOS Monterey 12.6.1\\Tensorflow\\Jupyter\end{tabular} \\ \hline
    \textit{\textbf{JUN SANG CHO}} & \begin{tabular}[c]{@{}l@{}}Windows 10\\TexLive 2022\end{tabular} \\ \hline
    \end{tabular}
    \renewcommand{\thetable}{\arabic{table}}
    \captionsetup{justification=centering}
\end{table}

\subsection{Cost Estimation}
To implement our application, it is necessary to obtain data from the database or to obtain real-time information from the server while communicating with the server in real-time. Therefore, real-time servers must be hosted and several APIs were needed.\\

\begin{table}[h]
    \caption{Cost Estimation}
    \centering
    \begin{tabular}{c|l}
    \hline
    \multicolumn{1}{l|}{\textit{\textbf{Tools and Language}}} & \textit{\textbf{Cost Estimation}} 
     & & \\ \hline
    \textit{\textbf{AWS EC2}} & \begin{tabular}[c]{@{}l@{}}AWS EC2 is a Virtual Private Server\\ (VPS) provider, which gives developers the\\ capacity and capabilities to compute, storage,\\ and networking to deploy and manage websites\\ and web applications in the cloud. AWS EC2\\ includes everything you need to get your project\\ up and running quickly (virtual machines,\\containers, databases, CDNs, load balancers, \\DNS management, etc.), and these services \\are available at low, predictable monthly rates. \end{tabular} \\ \hline
    \end{tabular}
    \renewcommand{\thetable}{\arabic{table}}
    \captionsetup{justification=centering}
\end{table}

\subsection{Software in Use}\label{SCM}
\begin{itemize}
\item [A.] Git \& Github\\
Git is a type of distributed version control system. Git records tasks within the project folder and enables systematic development through version management. Git allows multiple people to simultaneously develop the same files as one project without having to exchange source code separately. Github is a web hosting platform that supports projects using Git. Github is a cloud management version management system that provides a graphical user interface (GUI).\\
\\
\item [B.]Notion\\ 
Notion is an all-in-one productivity tool and collaboration tool that can efficiently create and manage notes, schedules, tasks, data, and projects. Notion is positioned as an all-in-one tool that replaces team wiki, project management, and document sharing tools as an enterprise collaboration tool. In addition, it is used for various purposes such as personal wikis, websites, company websites, blogs, and databases, and is known to be used for business in Korea as well as carrot markets and zigzag.\\
\\
\item [C.] Android Studio\\
Dementia Check is an application developed by collaboration of Ministry of Health and Welfare of Korea and the Central Dementia Center. It’s main functions include checking the risk of dementia, an encyclopedia of dementia, a function to find the elderly through GPS, a function to manage dementia people, a care diary function, facility information to help dementia patients, and messages containing hope. Dementia Check is currently available in both Android and iOS platforms.\\
\\
\item [D.] Xcode\\
Xcode is an IDE provided by Apple that can make various software for Apple such as macOS and iOS. Xcode is an application that runs only in macOS, and it is used to write objective-c or swift and develop applications. Xcode uses LLVM, which further improves the performance of GCC, as the main compiler.\\
\\
\item [E.] IntelliJ\\
IntelliJ IDEA is a commercial Java integrated development environment (IDE) produced by JetBrains. IntelliJ is an intelligent context-aware IDE for working with all kinds of applications such as Java and other JVM languages such as Kotlin, Scala, and Groovy.\\
\\
\item [F.] Visual Studio Code\\
Visual Studio Code is a source code editor developed by Microsoft as an open source. Built on the basis of electron from GitHub, it supports Microsoft Windows, macOS, and Linux. It also includes debugging support, Git control, syntax enhancement, SSH access, and allows users to modify the editor's themes, shortcuts, and settings.\\
\\
\item [G.] Postman\\
Postman is a platform that helps user to implement API development more quickly and easily, and helps test and document or share the developed API. Postman provides a variety of functions for API developers, including variables and environments, request descriptions, and script writing required for testing and pre-requests.\\
\\
\item [H.] AWS EC2(Elastic Compute Cloud)\\
EC2 is a cloud computing service provided by AWS. The service allows Amazon to remotely use the resources of the data center's server computers in each country. EC2 can increase or decrease capacity and pay as much as used, so it is economical. It also gives users complete control over their instances, and is effective in security, network configuration, and storage management.\\
\\
\item [I.] AWS RDS(Relational Database Service)\\
RDS is a relational database provided by AWS. RDS makes it easy to set up, operate, and scale relational databases in the cloud. RDS automates time-consuming management tasks such as database setup, hardware provisioning, patches, and backups, while providing cost-effective capacity for free sizing.\\
\\
\item [J.] Overleaf\\
Overleaf is a startup and social enterprise that builds modern collaborative authoring tools for scientists — like Google Docs for Science. Their primary product is an online, real time collaborative editor for papers, theses, technical reports and other documents written in the LaTeX markup language.
In their nine years since launch, Overleaf has seen rapid adoption across science and research, and now supports a community of over ten million authors from over 180 countries worldwide who’ve created over 100 million documents using our services, making users easily communicate and share their documents, idea, and errors. Overleaf does not require any installation, and provides various templates for user's convenience in documentation. Overleaf is very fast and can easily share documents with teammates, and fix errors simultaneously. Overleaf also provides real time screen of the documentation outcome in the right side of the page, allowing user respond to the changes they have made in the document immediately.\\
\\
\item [K.] TensorFlow\\
TensorFlow is an open-source software library for data flow programming for various tasks created by Google Brain team. It is a symbolic mathematical library and is also used in machine learning applications such as artificial neural networks and deep learning. TensorFlow offers APIs that facilitates Machine Learning. The goal is to implement this AI model in using NUGU speaker.\\
\\
\item [L.] NUGU playbuilder\\
NUGU playbuilder connects to NUGU speakers and supports a variety of services of NUGU, SK telecom’s artificial intelligence speaker. NUGU platform first identifies the intention of user utterance through voice recognition and natural language understanding. Then, it properly acts and responds through text-to-speech. NUGU playbuilder is a GUI based integrated development environment that offers techniques needed in this process.\\
\\
\end{itemize}

\section{SPECIFICATION}
\subsection{Loading Page}\label{AA}
This page utilizes the AppLoading function, which is one of a component of React Native. During the application loading time, both API data and Server data are loaded, and when the application finishes loading, it will automatically proceed to the next page. When the application is first launched, a background image with "MEMENTO," a white letter on a yellow background, is displayed to the user. Through this, the user can know that the application has been executed, and during that time, the program receives all the data necessary for the application.
\\
\subsection{Start Page}
This page is a page where it explains what role MEMENTO can play when the user first installs the application. This page is provided as a horizontal scroll view and consists of a total of five 
elements. By swiping the scroll view to the right, the user may check the overall description of the application.Pressing the login button from this page will lead to page C(Login Page).
\\
\subsection{Login Page}
This page is a page where all tasks related to the user's login are performed. In this page, user can login using their own Id and password. At the top of the screen, there is an input space where user can type Id and password. ID can be entered up to 8 characters, and password can be up to 10 characters. If both Id and password are entered correctly, the user can successfully login by pressing the login button located on the right. If user has logged in successfully, they will be redirected to page(x), Calendar page.\\
\\
\subsection{Self-Diagnosis}
User can conduct self-diagnosis about Youngzheimers whenever they want. This page provides a subjective memory complaints questionnaire (SMCQ) presented by the Central Dementia Center, an institution under the National Medical Center of Republic of Korea. The subjective memory loss questionnaire is a question to find out subjective memory and mood, and consists of questions about memory disorders that you usually experience subjectively. The user reads the corresponding questions and responds yes/no to what matches his/her behavior, thoughts, or feelings. For users that answered "Yes" to more than 6 out of a total of 14 questions, the result screen is displayed, "Danger! Please visit a nearby health center or dementia safety center for more accurate dementia examination.". For users who answered "Yes" to five or less questions, the result screen is displayed, "Safe! exercise well and participate various social activities, and try to follow the Dementia Prevention Rule 3.3.3 well to prevent dementia. If you want a more accurate dementia examination, please visit a nearby health center or dementia safety center.”. Upon completion of the diagnosis, user will be redirected to page(E), Recording Page.
\\
\subsection{Recording Page}
In this page, User is able to record "whenever, wherever daily", which is a key function of MEMENTO. This page consists of a microphone icon and a recording button in the middle of the page. The microphone icon helps the user to grasp at a glance what the function of this page is. The function of this page activates as soon as the user presses the button. When the user presses the button, the recording starts. As soon as recording begins, the recording button with a black mic icon will be converted into a black square. On this page, the user can record instantaneous emotions, appreciation, and others in daily activities. The recording stops if a user touches the black square button once again, and it will change its shape to initial state, a black mic icon button. The voice recording file in which the user speaks aloud is transmitted to its own server database along with the time information and location (GPS) information. The voice record spoken by the user is then converted into text based on STT technology. The text can be viewed at a glance in the form of a timeline in page(G), Daily Page.
\\
\subsection{Calendar Page(Main Page)}
This page consists of a calendar which a user can freely change the date. A monthly calendar is located in the center of a page, and just like any regular calendar, user can change month by touching an arrow icon both located on left or right of the calendar, and can move to another year by clicking the current year. Clicking current year will show a pop-up screen consisted of years, and user can choose an year they are looking for. After selecting year and month, user can select a certain day and move to the page(G), Daily Page. If a user had conducted more than one voice recording in page(E), Recording Page, their will be a timeline in page(G), Daily Page.And buttons are provided under the calendar to move to Self-Diagnosis Page(B), Prevention Game Page(H), and 
Prevention Information Page(I).
\\
\subsection{Daily Page}
This page consists of a timeline based on the user's daily activities. The function of a timeline is as follows. The server converts the user's voice recording file from page(E), Recording Page, into text through the STT technology-based React-Native library. After conversion finishes, it is stored in our database. It also provides users with additional timeline between activities when a user adds their LG home appliances in page(J) LG Appliances Page. This timeline is in the top-down direction, arranged in chronological order. Based on the location information stored in Recording Page (E), the Hanyang University logo is displayed on the timeline if it is recorded at Hanyang University, and the LG Electronics logo is displayed on the timeline if it is from LG Appliance Page (J). For other information, it is possible to provide users with criteria for distinguishing information by displaying the application character logo on the timeline.
\\
\subsection{Prevention Game Page}
This page consists of three games that help prevent Youngzheimer. The first game is to increase user's concentration. Users should say "red" instead of "blue" by looking at the letter "blue" written in  red color. Second game is to increase user's memory. User must choose a relevant category of a word shown to them in two seconds. After two seconds, a timeout function is executed and the screen is switched to a screen where a user has to select the correct answer. Whether the answer is correct or incorrect is notified to the user using the alert function. Third game is to increase user's computational power. A user will be given a simple eight-option arithmetic problem and provides the user with three questions per time.\\
\\
\subsection{Prevention Information Page}
This page provides information on the risk of dementia, the status of dementia, and prevention. It consists of a total of three categories. The first is a category that provides information on the risk of dementia. This area, which describes the risk of dementia provided by the Korea Centers for Disease Control and Prevention, can alert users to dementia by showing the risk of dementia. The second is a category that provides information on the status of dementia. Based on statistical data provided by the Central Dementia Center, an institution under the National Medical Center of Korea, the current status of dementia in Korea is provided. By checking this information, users can be alerted to prevent dementia in advance by seeing predictions that the expected population of dementia will double in 10, 20, and 30 years. The last is a category that provides information that can prevent dementia. This area, which provides vitamins and lifestyle habits that are good for preventing dementia, will help users prevent dementia.
\\
\subsection{LG Appliance Page}
This page is where a user can manage their LG home appliances added from page(K), Adding LG Appliance Page. Users can easily set their LG appliances and the days of each use. Through this, when the configuration button provided together is pressed, it is reflected along with the LG Electronics logo in the timeline of the Page(G), Daily Page. 
Through this page, users can operate LG home appliances without forgetting them 
and manage them effectively.
\\
\subsection{Adding LG Appliance Page}
This page is where a user can type serial number of their LG home appliances, choose the type of their LG home appliances, and choose what time and what day will the LG home appliances are planned to be used. The set LG home appliance information will be reflected in the LG Appliance Page (J) only when all elements are entered.
\\
\subsection{Link With NUGU}
MEMENTO helps prevent Youngzheimer by linking with SKT’s AI speaker, NUGU. As the user records daily impressions on page(E), Recording Page, this data is provided in the form of a timeline on page(G), daily page. NUGU extracts keywords from information based on this timeline and presents customized quizzes. Users can look back on their day by solving the daily quiz presented by NUGU. This can contribute to the improvement of the user's forgetfulness and the prevention of Youngzheimer through the process of reviewing their daily activities.
\\
\section{ARCHITECTURE DESIGN}
\subsection{Overall Architecture}

\begin{figure}[h]
\centerline{\includegraphics[width=9cm,height=9cm]{overall architecture final.png}}
\caption{Overall Architecture}
\label{fig}
\end{figure}
\\\\\\\\
Memento is largely composed of four modules: The frontend(app), server, AI and SKT NUGU speaker.\\\\
\indent The first module is frontend. We designed the MEMENTO application by using React Native to make it possible for the user to actually use the app themselves. User can check their potential possibility, or status of their threat in dementia or any other related diseases such as Alzheimer by conducting dementia check questions. Furthermore, users can play games that helps vitalizes brain activities, which are games that are officially proven by the central dementia center in Republic of Korea. In addition, users can record their daily activities whenever they want. Later, the recorded information and the location information will be saved as a timeline format to help user overview their daily life, and by going over their daily activities through quizzes with NUGU speaker will help user prevent Youngzheimer.\\
\\\\\\\\
\begin{figure}[h]
\centerline{\includegraphics[width=8cm,height=8cm]{database.jpg}}
\caption{Database}
\label{fig}
\end{figure}
\\
\indent The second module is backend which interacts with database. The backend of Memento is a database consisting of a main server implemented as a spring, a second server implemented as a flask, and database implemented as a MySQL. The main server has functions such as login, generating records and inquiry, generating quiz and inquiry, generating home appliances and inquiry, and daily activities data inquiry. Several data generated as a result of interaction with the user in the MEMENTO application are stored in database via spring server. In addition, data necessary for daily quiz questions and answers with users in NUGU speaker is inquired through spring server. In addition, the data necessary for generating user-recording-based daily quizzes are taken from the database, handed over to the second server, and the contents of the daily quizzes generated by the second server are stored again in the database. The second server has the function of generating a daily quiz. The machine learning model is executed using the recording data of the day received from the main server. The quizzes of the day generated as a result of the model execution are returned to the main server. Database implemented with mysql saves user information, appliance setting data, appliance running record, daily record data, quiz. All of these backend components are deployed on the cloud computing platform, Amazon Web Service (AWS). The main server and the second server are each implemented in separate AWS Compute Cloud (Elastical EC2), and the database is connected to AWS Relational Database Service (RDS).\\

\indent The third module is the AI. AI will serve as a role of storing datasets created from the information sent from the backend. There is an absolute lack of data on what users talk about at what time and what happened, and it is very difficult to obtain dataset that does not violate copyright. Our AI goes through the preprocessing step of removing unnecessary opening characters, removing left and right spaces in the answer, applying a tokenizer to modify data, removing fewer than six characters, and removing outliers. We have produced approximately 100 Q\&A labeling data for questions and answers using the haystack annotation tool on the entire dataset, and we can produce the six-way principle questions accordingly. Dataset augmentation was conducted through questions and answer labeling data, pytorch lighting dataset was used, and Q\&A data was pretained using KorQuAD 1.0 version, a Korean-only machine learning tool developed by Samsung SDS.\\

\indent The fourth module is NUGU playbuilder. This is a service provided by SKT, which allows developers to create services with NUGU and interact with it. NUGU serve as a role to enable users to use MEMENTO by communicating with their voice. Users can answer freely with their voices to the quizzes provided by NUGU speaker. \\
\\\\\\\\\\\\\\\\\\\\\\\\\\\\\\
\subsection{Directory Organization}
\\
\begin{figure}[h]
\centerline{\includegraphics[scale=0.43]{directory2.png}}
\caption{Repository}
\label{fig}
\end{figure}

\begin{table} [h]
    \caption{Directory Organization - Frontend 1}
    \centering
    \begin{tabular}{l|l|l}
    \hline
    \textit{\textbf{Directory}} & \textit{\textbf{File Name}} & \textit{\textbf{Library}}
     & & & \\ \hline
    \begin{tabular}[c]{@{}l@{}}MEMENTO-Men4/\\Frontend\end{tabular} & \begin{tabular}[c]{@{}l@{}}android\\assets/images\\ios\\navigation\\screen\\stack\\App.js\\README.md\\app.json\\atom.js\\babel.config.js\\images.js\\mystyled.js\\package-lock.json\\package.json\\styled.d.ts\\tsconfig.json\\yarn.lock\\\end{tabular} 
    & \begin{tabular}[c]{@{}l@{}}React\\React-Native\\expo-add-loading\\expo/vector-icons\\expo-asset\\@react-navigation/\\nativestyled-\\components/\\native\\recoil\end{tabular}\\ \hline
    \begin{tabular}[c]{@{}l@{}}MEMENTO-Men4/\\Frontend/assets/\\images/\end{tabular} & \begin{tabular}[c]{@{}l@{}}1024.png\\back1.png\\balloon.png\\bbiyak1.png\\bbiyak2.png\\bbiyak3.png\\bbiyak.png\\bbiyakbaksa.png\\bbiyakgame.png\\bbiyakgaming.png\\bbiyakhand.png\\bbiyakinfo.png\\bbiyaklove.png\\bbiyakloveibg.png\\bbiyakmemory.png\\bbiyaknugu.png\\current1\\current2.png\\current3.png\\current4.png\\hanyang.png\\LG.png\\LGstart.png\\logo\_J.png\\prevent1.png\\prevent2.png\\Record.png\end{tabular} 
    & \begin{tabular}[c]{@{}l@{}}\end{tabular}\\ \hline
    \begin{tabular}[c]{@{}l@{}}\\MEMENTO-Men4/\\Frontend/ios/\end{tabular} & \begin{tabular}[c]{@{}l@{}}memento.xcodeproj\\memento.\\xcworkspace\\memento\\Podfile\\Podfile.lock\\AppDelegate.h\\Info.plist\\\end{tabular}
    & \begin{tabular}[c]{@{}l@{}}$<$React/RCTBridge.h$>$\\$<$ReactRCTBundle-\\URLProvider.h$>$\\$<$React/RCTRootView.h$>$\\$<$React/RCTLinking-\\Manager.h$>$\\$<$React/RCTConvert.h$>$\end{tabular}\\ \hline
    \begin{tabular}[c]{@{}l@{}}MEMENTO-Men4/\\Frontend/navigation/\end{tabular} & \begin{tabular}[c]{@{}l@{}}Drawer.js\\Root.js\\Stack.jsx\\Tabs.js\\\end{tabular}
    & \begin{tabular}[c]{@{}l@{}}React\\React-Native\\styled-components/native\\@expo/vector-icons\\@react-navigation/drawer\\@react-navigation/native\\react-native-timeline-flatlist\\@react-navigation/bottom\\-tabs\\recoil \end{tabular}\\ \hline
    \begin{tabular}[c]{@{}l@{}}MEMENTO-Men4/\\Frontend/stack/\end{tabular} & \begin{tabular}[c]{@{}l@{}}Days.js\\Answer.js\\Game.js\\Game1.js\\Game2.js\\Game3.js\\Start.js\end{tabular} 
    & \begin{tabular}[c]{@{}l@{}}React\\React-Native\\styled-components/native\\@react-navigation/native\\recoil \end{tabular}\\ \hline
\end{tabular}
\end{table}

\begin{table} [h]
    \caption{Directory Organization - Frontend 2}
    \centering
    \begin{tabular}{l|l|l}
    \hline
    \textit{\textbf{Directory}} & \textit{\textbf{File Name}} & \textit{\textbf{Library}}
     & & & \\ \hline
    \begin{tabular}[c]{@{}l@{}}\\MEMENTO-Men4/\\Frontend/screens/\end{tabular} & \begin{tabular}[c]{@{}l@{}}\\home\\Calendar.js\\Diagnosis.jsx\\LG.jsx\\Login.js\\Recording.jsx\\Write.jsx\\\end{tabular}
    & \begin{tabular}[c]{@{}l@{}}React\\React-Native\\styled-components/native\\@expo/vector-icons\\@react-navigation/native\\react-native-bouncy-\\checkbox\\react-native-highlight-\\underline-text\\react-native-calendars\\axios\\react-native-\\modal-datetime-picker\\date-fns\\date-fns/esm/locale/ko/\\react-native-gesture-\\handler\\recoil\\react-native-voice \end{tabular}\\ \hline
    \end{tabular}
\end{table}

\begin{table} [h]
    \centering
    \caption{Directory Organization - Backend 1}
    \begin{tabular}{l|l|l}
    \hline
    \textit{\textbf{Directory}} & \textit{\textbf{File Name}} & \textit{\textbf{Library}}
     & & & \\ \hline
     \begin{tabular}[c]{@{}l@{}}MEMENTO-Men4/\\Backend/\\\end{tabular} & \begin{tabular}[c]{@{}l@{}}src/main\\build.gradle\\\end{tabular}
    & \begin{tabular}[c]{@{}l@{}}\end{tabular}\\ \hline
    \begin{tabular}[c]{@{}l@{}}MEMENTO-Men4/\\Backend/src/main/\\\end{tabular} & \begin{tabular}[c]{@{}l@{}}java/hyu\_memento/\\memento\_back\\resources\\\end{tabular} 
    & \begin{tabular}[c]{@{}l@{}}\end{tabular}\\ \hline
    \begin{tabular}[c]{@{}l@{}}MEMENTO-Men4/\\Backend/src/main/\\resources/\end{tabular} & \begin{tabular}[c]{@{}l@{}}application.yml\\\end{tabular} 
    & \begin{tabular}[c]{@{}l@{}}\end{tabular}\\ \hline
    \begin{tabular}[c]{@{}l@{}}MEMENTO-Men4/\\Backend/src/main/\\java/hyu\_memento/\\memento\_back/\end{tabular} & \begin{tabular}[c]{@{}l@{}}\\controller\\domain\\repository\\service\\\end{tabular} 
    & \begin{tabular}[c]{@{}l@{}} \end{tabular}\\ \hline
    \begin{tabular}[c]{@{}l@{}}\\MEMENTO-Men4/\\Backend/src/main/\\java/hyu\_memento/\\memento\_back/\\controller/\end{tabular} & \begin{tabular}[c]{@{}l@{}}dto\\ApplianceController.java\\ApplianceOperation\\Controller.java\\GameplayController.java\\MemberController.java\\QuizController.java\\RecordController.java\\\end{tabular}
    & \begin{tabular}[c]{@{}l@{}}JSONObject\\DateTimeFormat\\RequiredArgs-\\Constructor\\Getter\\LocalDate\\LocalTime\\DateTime-\\Formatter\\ArrayList\\List\end{tabular}\\ \hline
    \end{tabular}
\end{table}

\begin{table} [h]
    \centering
    \caption{Directory Organization - Backend 2}
    \begin{tabular}{l|l|l}
    \hline
    \textit{\textbf{Directory}} & \textit{\textbf{File Name}} & \textit{\textbf{Library}}
     & & & \\ \hline
    \begin{tabular}[c]{@{}l@{}}MEMENTO-Men4/\\Backend/src/main/\\java/hyu\_memento/\\memento\_back/\\controller/dto/\end{tabular} & \begin{tabular}[c]{@{}l@{}}ApplianceReturnDto.java\\ApplianceSaveDto.java\\ApplianceReturnDto.java\\ApplianceSaveDto.java\\GameReturnDto.java\\GameSaveDto.java\\MemberDto.java\\QuizDto.java\\RecordReturnDto.java\\RecordSaveDto.java\\\end{tabular}
    & \begin{tabular}[c]{@{}l@{}}Builder\\Getter\\NoArgs-\\Constructor\\ \end{tabular}\\ \hline
    \begin{tabular}[c]{@{}l@{}}MEMENTO-Men4/\\Backend/src/main/\\java/hyu\_memento/\\memento\_back/\\controller/nugu\_dto/\end{tabular} & \begin{tabular}[c]{@{}l@{}}NuguOutputdto.java\\NuguReturnDto.java\\\end{tabular}
    & \begin{tabular}[c]{@{}l@{}}Builder\\Getter\\NoArgs-\\Constructor \end{tabular}\\ \hline
    \begin{tabular}[c]{@{}l@{}}\\MEMENTO-Men4/\\Backend/src/main/\\java/hyu\_memento/\\memento\_back/domain/\end{tabular} & \begin{tabular}[c]{@{}l@{}}type\\Appliance.java\\ApplianceOperation.java\\GamePlay.java\\Member.java\\Quiz.java\\QuizContent.java\\Record.java\\\end{tabular}
    & \begin{tabular}[c]{@{}l@{}}Builder\\Getter\\NoArgs-\\Constructor\\LocalTime\\LocalDate\\ArrayList\\List\end{tabular}\\ \hline
    \begin{tabular}[c]{@{}l@{}}\\MEMENTO-Men4/\\Backend/src/main/\\java/hyu\_memento/\\memento\_back/domain/\\type/\end{tabular} & \begin{tabular}[c]{@{}l@{}}ApplianceDayStatus.java\\ApplianceType.java\\GameType.java\\Gender.java\\MemberType.java\end{tabular}
    & \begin{tabular}[c]{@{}l@{}}\end{tabular}\\ \hline
    \begin{tabular}[c]{@{}l@{}}\\MEMENTO-Men4/\\Backend/src/main/\\java/hyu\_memento/\\memento\_back/repository/\end{tabular} & \begin{tabular}[c]{@{}l@{}}ApplianceOperation\\Repository.java\\ApplianceRepository.java\\GameplayRepository.java\\MemberRepository.java\\QuizRepository.java\\RecordRepository.java\\\end{tabular}
    & \begin{tabular}[c]{@{}l@{}}Constructor\\Repository\\EntityManager\\Persistence-\\Context\\LocalDate\\DayOfWeek\\List\end{tabular}\\ \hline
    \begin{tabular}[c]{@{}l@{}}\\MEMENTO-Men4/\\Backend/src/main/\\java/hyu\_memento/\\memento\_back/service/\end{tabular} & \begin{tabular}[c]{@{}l@{}}ApplianceOperation\\Service.java\\ApplianceService.java\\GameplayService.java\\MemberService.java\\QuizService.java\\RecordService.java\\\end{tabular}
    & \begin{tabular}[c]{@{}l@{}}RequiredArgs-\\Constructor\\Service\\Transactional\\LocalDate\\ArrayList\\List\end{tabular}\\ \hline
    \begin{tabular}[c]{@{}l@{}}\\MEMENTO-Men4/\\Backend/build\end{tabular} & \begin{tabular}[c]{@{}l@{}}ApplianceOperation\\Service.java\\ApplianceService.java\\GameplayService.java\\MemberService.java\\QuizService.java\\RecordService.java\\\end{tabular}
    & \begin{tabular}[c]{@{}l@{}}\end{tabular}\\ \hline
    \begin{tabular}[c]{@{}l@{}}MEMENTO-Men4/\\Backend-for-Ai\\\end{tabular} & \begin{tabular}[c]{@{}l@{}}app.py\\best-checkpoint-v6.ckpt\\\end{tabular}
    & \begin{tabular}[c]{@{}l@{}}json\\pandas\\numpy\\tokenizer\\torch\\pytorch\_\\lightning\\transformers\\flask\end{tabular}\\ \hline
    \end{tabular}
\end{table}

\begin{table} [h]
    \centering
    \caption{Directory Organization - AI 1}
    \begin{tabular}{l|l|l}
    \hline
    \textit{\textbf{Directory}} & \textit{\textbf{File Name}} & \textit{\textbf{Library}}
     & & & \\ \hline
    \begin{tabular}[c]{@{}l@{}}MEMENTO-Men4/\\AI-recognition\\\end{tabular} & \begin{tabular}[c]{@{}l@{}}data\_augmentation\\data\_utils\\model\\preprocessing\\README.md\\\end{tabular} 
    & \begin{tabular}[c]{@{}l@{}}\end{tabular}\\ \hline
    \begin{tabular}[c]{@{}l@{}}MEMENTO-Men4/\\AI-recognition/\\kowiki/\end{tabular} & \begin{tabular}[c]{@{}l@{}}ko\_32000.model\\ko\_32000.vocab\\my\_corpus.txt\\nanumbarungothic.ttf\\ratings\_test.txt\\ratings\_train.txt\\\end{tabular} 
    & \begin{tabular}[c]{@{}l@{}}os\\re\\numpy\\pandas\\pickle\\random\\collection\\json \\ \end{tabular}\\ \hline
    \begin{tabular}[c]{@{}l@{}}MEMENTO-Men4/\\AI-recognition/\\data/\end{tabular} & \begin{tabular}[c]{@{}l@{}}memento\\pretrain\_data\\qa\_data\\KorQuAD\_EDA.ipynb\\KorQuAD\_test\_df.csv\\KorQuAD\_train\_df.csv.zip\\vocab.txt\\\end{tabular} 
    & \begin{tabular}[c]{@{}l@{}}tensorflow-\\addons\\sentencepiece\\transformers\\keras\\absolute\_import\\division\\print\_function\\unicode\_liteals\\os\\re\\numpy\\pandas\\pickle\\random\\collection\\json\\datetime\\tqdm\\matplotlib\\seaborn\\wordcloud \end{tabular}\\ \hline
    \begin{tabular}[c]{@{}l@{}}MEMENTO-Men4/\\AI-recognition/\\data/memento\end{tabular} & \begin{tabular}[c]{@{}l@{}}Memento\_date\_append.ipynb\\overall\_data.csv\\test\_data.csv\\test\_data.json\\train\_data.csv\\train\_data.json\\\end{tabular} 
    & \begin{tabular}[c]{@{}l@{}}os\\re\\numpy\\pandas\\pickle\\random\\collections\\json\\datetime\\ \end{tabular}\\ \hline
    \begin{tabular}[c]{@{}l@{}}MEMENTO-Men4/\\AI-recognition/\\data/pretrain\_data\end{tabular} & \begin{tabular}[c]{@{}l@{}}all.json\\math\_ko.json\\transfered\_data.pkl\\\end{tabular} 
    & \begin{tabular}[c]{@{}l@{}}\\ \end{tabular}\\ \hline
    \begin{tabular}[c]{@{}l@{}}MEMENTO-Men4/\\AI-recognition/\\data/qa\_data\end{tabular} & \begin{tabular}[c]{@{}l@{}}qa\_data\_test.json\\qa\_data\_train.json\\\end{tabular} 
    & \begin{tabular}[c]{@{}l@{}}\\ \end{tabular}\\ \hline
    \begin{tabular}[c]{@{}l@{}}MEMENTO-Men4/\\AI-recognition/\\model/\end{tabular} & \begin{tabular}[c]{@{}l@{}}KorQuAD\\Q\&A\\bert\\elmo\\\end{tabular} 
    & \begin{tabular}[c]{@{}l@{}}\\ \end{tabular}\\ \hline
    \end{tabular}
\end{table}

\begin{table} [h]
    \centering
    \caption{Directory Organization - AI 2}
    \begin{tabular}{l|l|l}
    \hline
    \textit{\textbf{Directory}} & \textit{\textbf{File Name}} & \textit{\textbf{Library}}
     & & & \\ \hline
     \begin{tabular}[c]{@{}l@{}}MEMENTO-Men4/\\AI-recognition/\\model/KorQuAD\end{tabular} & \begin{tabular}[c]{@{}l@{}}KorQuAD\_v1.0\_dev.json\\KorQuAD\_v1.0\_train.json.zip\\README.md\\evaluate\_v1\_0.py\\requirements.txt\\run\_squad.py\\tokeniztion\_kobert.py\\\end{tabular} 
    & \begin{tabular}[c]{@{}l@{}}print\_\\function\\Counter\\string\\re\\argparse\\json\\sys\\os\\argparse\\glob\\logging\\random\\timeit\\numpy\\torch\\util.date\\utils.\\distributed\\DateLoader\\Random-\\Sampler\\Sequential-\\Sampler\\Distributed-\\Sampler\\tqdm\\trange\\PreTrained-\\Tokenizer \end{tabular}\\ \hline
    \begin{tabular}[c]{@{}l@{}}MEMENTO-Men4/\\AI-recognition/\\model/Q\&A/\end{tabular} & \begin{tabular}[c]{@{}l@{}}arguments.py\\elasticsearch.py\\inference.py\\qa\_train.py\\realtime\_model\_1.ipynb\\realtime\_model\_2.ipynb\\retrieval.py\\submission.py\\sweep.yaml\\train.py\\trainer\_qa.py\\\end{tabular} 
    & \begin{tabular}[c]{@{}l@{}}sys\\dataclass\\field\\Optional\\json\\pprint\\warnings\\re\\os\\argparse\\tqdm\\Elasticsearch\\Callable\\Dict\\List\\NoReturn\\Tuple\\numpy\\streamlit\\Dataset\\DatasetDict\\Features\\Value\\load_matric\\AutoConfig\\AutoModelFor\\QuestionAnswering\\AutoTokenizer\\DataCollator\\WithPadding\\EvalPrediction\\HfArgument\\Parser\\TrainingArguments\\pandas\\DataTraining\\Arguments\\ModelArguments\\SparsRetrieval\\ElastricRetrieval\\Question\\AnsweringTrainer\\postprocess\_qa\_\\preditions\\AutoTokenizer \end{tabular}\\ \hline
    \end{tabular}
\end{table}

\begin{table} [h]
    \setlength{\tabcolsep}{1pt}.
    \centering
    \caption{Directory Organization - AI 3}
    \begin{tabular}{l|l|l}
    \hline
    \textit{\textbf{Directory}} & \textit{\textbf{File Name}} & \textit{\textbf{Library}}
     & & & \\ \hline
     \begin{tabular}[c]{@{}l@{}}MEMENTO-Men4/\\AI-recognition/\\model/Q\&A/\end{tabular} & \begin{tabular}[c]{@{}l@{}}arguments.py\\elasticsearch.py\\inference.py\\qa\_train.py\\realtime\_model\_1.ipynb\\realtime\_model\_2.ipynb\\retrieval.py\\submission.py\\sweep.yaml\\train.py\\trainer\_qa.py\\\end{tabular} 
    & \begin{tabular}[c]{@{}l@{}}AutoModelFor\\QuestionAnswering\\load\_dataset\\load\_metric\\collections\\torch\\TrainingArguments\\Trainer\\default\_data\_collator\\EarlyStoppingCallback\\tqdm\\path\end{tabular}\\ \hline
    \begin{tabular}[c]{@{}l@{}}MEMENTO-Men4/\\AI-recognition/\\model/bert/\end{tabular} & \begin{tabular}[c]{@{}l@{}}create\_pretraining\_data.py\\modeling.py\\optimization.py\\run\_pretraining.py\\tokenization.py\\\end{tabular} 
    & \begin{tabular}[c]{@{}l@{}}absolute_import\\division\\print_function\\collections\\random\\sys\\tokenization\\tensorflow\\unicodedata\\six\\re\end{tabular}\\ \hline
    \begin{tabular}[c]{@{}l@{}}MEMENTO-Men4/\\AI-recognition/\\model/elmo/\end{tabular} & \begin{tabular}[c]{@{}l@{}}README.md\\elmo-vocab.txt\\options.json\\\end{tabular} 
    & \begin{tabular}[c]{@{}l@{}}train\\load\_options\_\\latest\_checkpoint\\load\_vocab\\BidirectionalLMDataset\\argparse \end{tabular}\\ \hline
    \begin{tabular}[c]{@{}l@{}}MEMENTO-Men4/\\AI-recognition/\\preprocessing/\end{tabular} & \begin{tabular}[c]{@{}l@{}}\_\_init\_\_.py\\dump.py\\mecab-user-dic.csv\\supervised\_nlputils.py\\unsupervised\_nlputils.py\\haystack\_preprocessing\_\\new.ipynb\\insert\_\\data\_preprocessing.py\\memento\_data\_\\preprocessing.ipynb\\preprocess.sh\\preprocessing\_question\\\&answering\_\\new.ipynb\\\end{tabular} 
    & \begin{tabular}[c]{@{}l@{}}re\\json\\glob\\argparse\\WikiCorpus\\Dictionary\\to\_unicode\\sys\\KhaiiiAPi\\Okt\\Komoran\\Mecab\\Hannanum\\Kkma\\math\\argparse\\WordExtractor\\LTokenizer\\CountSpace\\character\_is\_korean\\decompose\\spm\\FullTokenizer\\convert\_to\_unicode\\pandas \end{tabular}\\ \hline
 \end{tabular}
\end{table}

\clearpage
\subsection{Module 1: Frontend}
\begin{enumerate}
    \item Purpose\\
    To develop MEMENTO, we used React Native so that a it can be used in a cross-platform for both Android and iOS environment. Frontend serves as a role in providing interface to a user. It connects user and server by providing input fields that a user can enter information. Once this job is done, the values are delivered to backend. Also, Frontend brings data in the database and provides it to the user after changing it to a data that a user can understand.\\ \\
    \item Functionality\\
    Creating an account by entering email, password, ID, or using social account, conducting dementia check questions, making recordings, playing dementia prevention games, going over calendar, going over daily timeline, checking useful information that helps prevent dementia. Frontend shows information obtained from database, by making a request to backend.\\ \\
    \item Location of source code\\: www.github.com/MEMENTO-Men4/Frontend \\ \\
    \item Class component\\
        \item[-] ios folder : This is a folder containing a file that replaces the JavaScript-written React native code with the iOS-only swift code.\\
        \item[-] memento.xcodeproj : This is a file that is originally a React-Native code written in JavaScript, and later converted to a swift format so that it can be used in iOS environment. \\
        \item[-] memento.xcworkspace: xcodeproj : This is a file that contains all the required options that a xcodeproj file needs inside a iOS emulator.\\ 
        \item[-] memento folder: This is a folder in which the three reflected files are stored in the screen navigation components provided by React Native.\\
        \item[-] Podfile : This is a Ruby-written file, the language required to run React Native in an Xcode emulator, is a set of instructions describing the dependencies of one or more Xcode project targets. The 'Pod install' command refers to the contents of the pod file to install the necessary libraries in the pod directory. \\ 
        \item[-] navigation folder : This is a folder in which the three reflected files are stored in the screen navigation components provided by React Native.\\
        \item[-] Root.jsx : This file serves as the node used to connect each js file. It consists of Drawer.jsx and Stack.jsx. Use a navigation module called 'createNativeStackNavigator' to enable stack navigation. This module object is declared Nav and used as a custom component, which acts as an interconnect between Drawer.jsx and Stack.jsx. \\
        \item[-] Drawer.jsx : This is a JavaScript file for providing menu options to users. Clicking on the upper left menu icon or swiping the left screen is exposed to the user through the "react-native-gesture-handler" module. It distinguishes whether or not it is logged in through the value of the 'loginFlag' variable that acts as a global variable. Drawer.jsx is classified into a home screen, a login/logout screen, and a Youngzheimers self-diagnosis screen, and JavaScript files for each feature are stored in the screens folder. \\
        \item[-] Tabs.jsx : This is a JavaScript file for providing Tab Bar located at the bottom of the home screen. The Tab Bar, visible only on Drawer's home screen, consists of a calendar (home), voice recording, LG home appliances and game screens. Clicking the icon on the tab bar will move you to the appropriate screen for each icon. \\
        \item[-] Stack.jsx : This is a JavaScript file for providing a stack navigation effect to a user. On the screen that makes up Drawer.js and Tabs.jsx, clicking a specific button takes you to the screen of the file that makes up Stack.jsx. At this time, the moving effect is a 'stack'-like animation that accumulates from right to left. To apply this stack animation when moving to a screen, it must be declared a Screen component in this Stack.jsx file. \\
        \item[-] screens folder : This folder is a collection of files that make up the application screen, which make up the navigation files declared in the navigation folder. \\
        \item[-] Calendar.jsx : This is a file that constitutes the first screen that the user sees when the application is executed. The screen consists of a calendar created using an open-source library called "react-native-calenders" and buttons that allow you to move on to the Youngzheimers information introduction, Youngzheimers self-diagnosis, and Youngzheimers prevention game screen. \\
        \item[-] Diagnosis.jsx : This is a file provided to check whether the user himself/herself has suspected symptoms of Youngzheimer's. It consists of 12 questionnaire questions and 12 check boxes, and when you press the submit button, the alert notifies the user whether or not it is Youngzheimers. Accessible from the left menu and calendar (home) screen.  \\
        \item[-] LG.jsx : This is a JavaScript file with LG screen that allows users to set the time and duration after completing the registration of LG home appliances they own.\\
        \item[-] Write.jsx : This file is a screen that registers LG home appliances owned by users. You can set the serial number, home appliance type, operating day, and time.\\
        \item[-] Login.jsx : This file serves as a role when a user enters an ID, e-mail, or password, it performs verification by computing it, executes login if all information matches, and restricts login if the information does not match.\\
        \item [-] Recording.jsx : This file is a screen that allows users to record. Using an open-source library called "react-native-voice", we applied a technology that converts the voice recording file into text after the user makes a voice recording.\\
        \item[-] stack folder : This is a folder containing files that apply 'stack' animation. These folders are subdivided into 'calendarDay', 'game' and 'info' folders, respectively.\\
        \item[-] calendarDay folder : This is a folder in which the file that constitutes the calendar's functionality in the application resides.\\
        \item[-] Days.jsx : This is a JavaScript file that constitutes a screen that is rendered when a specific date of the calendar is clicked. It was implemented using axios to communicate with the server so that each date could have different information.\\
        \item[-] game folder : This is a folder where are the files that make up three different types of games that are known to be helpful in preventing dementia. \\
        \item[-] Game1.jsx : This is a file that performs a dementia prevention game by representing a random colored word as a pop-up, and the word will refer to a word of a different color from the actual color of the word. Users should choose "red" instead of "black" by looking at the letter "black" written in  red color. The main goal of this file is to provide user a game that increases user's concentration. \\
        \item[-] Game2.jsx : This is a file that performs a memory game in which a word is shown to the user and the type that is most relevant to the word has to be selected in two seconds. After two seconds, a timeout function is executed and the screen (Answer.jsx) is switched to a screen where a user has to select the correct answer. Whether the answer is correct or incorrect is notified to the user using the alert function. The main goal of this file is to provide user a game that increases user's memory.\\
        \item[-] Game3.jsx : This is a file that performs a game that shows users a simple four-step arithmetic problem to increase their computational power by selecting the correct answer, and provides the user with three questions per time. The correct or incorrect answer is notified to the user using the alert function. The main goal of this file is to provide user a game that increases user's calculation ability.\\
        \item[-] info folder : This is a folder that contains all information related to dementia that a user can access in the application, and is subdivided into the risk of dementia, the overall status of dementia, and methods how to prevent dementia.\\
        \item[-] Infos.jsx : This is a JavaScript file that constitutes a screen that is rendered when you click the upper left menu icon or when you click a button on the calendar (home) screen. It consists of three buttons, and each button is implemented to select one of the three information.\\
        \item[-] Risk.jsx : This is a JavaScript file that provides information to help users know the risk of dementia.\\
        \item[-] CurrentSituation.jsx : This is a JavaScript file that provides information for users to know the status of dementia.\\
        \item [-] Prevent.jsx : This is a JavaScript file that provides information to help users know how to prevent dementia.\\
        \item [-] App.jsx : This is the top-level JavaScript file of all the files that make up the application. Recoil for using global variables, dark mode themes, fonts, and other necessary configurations for running applications are gathered.\\
        \item [-] atom.jsx : This is a JavaScript file that declares a global variable that can be used anywhere in the entire JavaScript file. The value of a variable declared through atom function in this file can be retrieved from another JavaScript file via the react hook 'useRecoilState'.\\
        \item [-] mystyled.js : This is a color code created through JavaScript so that dark mode and light mode can be selected to suit the user's taste. \\
        \item [-] start.jsx : This file is consists of five pages that describes what role a MEMENTO can play. Pressing login button from any of the description page leads user to login page.
\end{enumerate}

\subsection{Module 2: Backend}
\begin{enumerate}
    \item Purpose\\
    The backend is responsible for managing servers and databases. The backend stores and manages data, and handles actions taken by users on the client-side of the application. The backend is responsible for putting the data generated as a result of the user's behavior in db at the front desk and inquiring the data required by the user from db at the front desk. We used spring and flask to implement Memento's backend. We adopted Spring as the language to implement the main server because it is a Java-based framework and also an e-government standard framework recommended for use in the development of web services for public institutions in Korea. In the memento project architecture, the spring has two clients. The first is the reaction native instance of Memento. The spring server processes the request of the react native app and returns an appropriate response. The second is SKT's NUGU ai speaker instance. Spring server returns to NUGU a list of quizzes needed to proceed with daily quizzes, a key function of memento. We choose Flask to implement our second server because Flask is a python-based framework that is good for serving a python-based ml model and is simpler than other python frameworks, Django.\\ \\
    \item Functionality\\
    Memento performs daily recording through the recording function of the mobile app and stores it in the database of the spring server(main server-frontend). Based on this daily recording data, quiz data is created and stored in database. These quizzes are delivered to NUGU, and the user conducts daily quizzes with artificial intelligence speakers (main server-NUGU). Main server's functionality is member creation, recording generation and inquiry, quiz generation and inquiry, home appliance generation and inquiry, daily data inquiry. Second server's functionality: quiz generation through machine learning.\\\\
    \item Location of source code\\Main Server : www.github.com/MEMENTO-Men4/Backend/\\
    Second Server : www.github.com/MEMENTO-Men4/Backend-for-Ai/ \\ \\
    \item Class component
    \item[] Main Server(MEMENTO-Men4/Backend/)\\
        \item[-] build.gradle : This is a file that contains a set of plugins and dependencies required for spring to run properly.\\
        \item[-] src/resources/application.yml : This is a file that manages settings used in spring projects.\\
        \item[-] src/main/java/hyu memento/memento back/controller : This ia a folder that contains files that process user requests and then hand over model objects to the specified view.\\
        \item [-] ApplianceController : This is a file that contains API information related to LG home appliances.\\
        \item [-] ApplianceOperationController : This is a file that contains API information related to operation records of LG home appliances.\\
        \item [-] GameplayController : This is a file that contains API information related to the game execution history.\\
        \item[-] MemberController : This is a file that contains API information related with managing users.\\
        \item [-] QuizController : This is a file that contains API information related with NUGU quizzes.\\
        \item [-] RecordController : This is a file that contains API information related with recording function.\\
        \item [-] DayController : This is an API for Daily Page.\\
        - src/main/java/hyu memento/memento back/controller/dto/ : This is folder that contains all the Data Transfer Object files.
        \item [-]ApplianceReturnDto : This is a class that contains values to be returned when looking up a list of appliances.\\
        \item[-] ApplianceSaveDto : This is a class of json mapped to object from frontend to save home appliances.\\
        \item [-] GameReturnDto : This is a class that contains values to be returned when inquiring game records.\\
        \item [-] GameSaveDto : This is a class of json mapped to the object and handed over from the frontend to store game execution history.\\
        \item [-] MemberDto : This is a class that maps and contains json handed over from the frontend when the ID was created after registering from the registration function.\\
        \item [-] RecordReturnDto : This is a class that contains the values to be returned when checking recording records.\\
        \item [-] RecordSaveDto : This is a class of json mapped to the object and handed over from the front end to store recordings.\\
        \item [-] FlaskResponseDto : This is a dto to get the quiz contents generated from the flask server, which is the second server\\
        \item [-] DayDto : This is a dto containing recording or home appliance data required for the day page of memento.\\
        \item [-] src/main/java/hyumemento/mementoback/controller/nugu\_dto : This is a dto used for main server and communication between NUGU.
        \item [-] NuguReturnDto : This is a dto to be returned to NUGU. It contains the version of bakcend proxy server, resultCode, and output.\\
        \item [-] NuguOuputDto : This is a dto with key contents to be returned to nugu.\\
        - src/main/java/hyu memento/memento back/service/ : This is a folder that contains all files that handle business logic and transactions.\\
        \item [-] src/main/java/hyu memento/memento back/domain/ : This is a folder that contains all the entity files.\\
        \item [-] Appliance : This is a file for the LG home appliances.\\
        \item [-] ApplianceOperation : This is a file for operation records of LG home appliances.\\
        \item [-] GamePlay : This is a file for the execution history of dementia prevention game.\\
        \item [-] Member : This is a file for users.\\
        \item [-] Quiz : This is a file for Quizzes.\\
        \item [-] QuizContent : This is a file that contains all the information of the Quizzes.\\
        \item [-] Record : This is a file for recordings.\\
        \item [-] src/main/java/hyu memento/memento back/domain/type : This is a folder that contain all the enums to be used in the domain.\\
        \item [-] ApplianceType : This is a File that contains home appliances such as WASHING MACHINE, DISH MACHINE, CLOTH DRYER, STYLER, AIR CLEANER, WATER MACHINE.\\
        \item [-] GameType : This is file that contains the types of dementia prevention games such as MATH (simple number calculation game), COLOR (color reverse reading), REVERSE (letter reverse reading), etc.\\
        \item [-] Gender : This is a file for gender classification of MALE (male), FEMALE (female) members.\\
        \item [-] MemberType : This is file for user type classification such as ADMIN(Administrator) and GENERAL(General Users).\\
        \item [-] src/main/java/hyu memento/memento back/repository : This is a folder that contains files that use JPA to access database.\\
        \item[] Second Server(MEMENTO-Men4/Backend-for-Ai//)\\
        \item [-] app.py : This is a flask code that receives data from the request, executes machine learning, and sends the result value as a response.\\
        \item [-] best-checkpoint-v6.ckpt : It is a model manufactured by pretraining with KorQuAD 1.0 and finetuning using RoBERTa, and returns questions according to the user's utterance.\\
        \item [-] .gitattributes : best-checkpoint-v6.ckpt capacity is more than 100mb, so upload using Git LFS. .gitattributes is a file that manages information about files that you track with lfs.\\
\end{enumerate}

\subsection{Module 3: AI}
\begin{enumerate}
    \item Purpose\\
    There was an absolute lack of data on the timeline where users talked lightly about what had happened in their daily lives, and there was no enough timeline datasets that were free from copyright issues, so we have created our own dataset consisted of approximately 1,500 sentences that contains time, location, name of a person, incidents, and emotion.  \\ \\
    \item Functionality\\
    It removes unnecessary opening characters, removes left and right spaces in your answer, modifies data after applying Tokenizer, removed all words consisted of less than 6 characters, removes outliers. After this process, Q\&A Labeling data for questionnaires and answers using the Haystack annotation tool across the entire Datasets are created. \\ \\
    \item Location of source code\\: www.github.com/MEMENTO-Men4/Ai-recognition/ \\ \\
    \item Class component\\
        \item[-] Readme.md : This is a file that contains all the content related to artificial intelligence used in the project
        \item[-] data\_augmentation folder : This is a folder that contains all the codes used to expand data and codes that analyze what data should be expanded, especially through EDA.\\
        \item[-] data\_utils folder : This is a folders that stores all the processes that make data available, such as splitting and uploading data into train sets and test sets, and refining data using haystack annotation tools.\\
        \item[-] model folder : This is a folder that contains codes for all models used, such as the elastic search engine used as the retriever model and RoBERTa used as the retrieval model.\\
        \item[-] memento.xcodeproj : This is a file that is originally a React-Native code written in JavaScript, and later converted to a swift format so that it can be used in iOS environment.\\
        \item[-] : memento.xcworkspace: xcodeproj : This is a file that contains all the required options that a xcodeproj file needs inside a iOS emulator.\\
        \item[-] memento folder : This is a folder that contains the python code Info.plist which causes the settings for the API added in React Native to be converted to xcode settings.\\
        \item[-] preprocessing folder : This is a folder that contains files that preprocess data produced using Haystack's annotation tool, files that allow immediate preprocessing when data is inserted, files that focus on unsupervised learning in Korean, and files that preprocess through supervised learning, preprocess questioning and answering.\\
        \item [-] kowiki folder : This is a folder that contains the model that learned Korean vocabulary through Word2Vec using Korean wiki text data and Korean vocabularies.\\
        \item [-] data/KorQuAD\_EDA.ipynb : This is a a foler in which graphs are stored by analyzing the results, evaluation, and statistical data of the KorQuAD model.\\
        \item [-] model folder : This is a folder that contains codes for all models used, such as the elastic search search engine used as the retriever model and the RoBERTa model used as the reader model.\\
        \item [-] model/KorQuAD folder : This is a folder that contains datasets for reading Korean machines that build extractive MRC data for Korean Wikipedia, calculate the similarity between words and sentences, and enable it to be used as input from other models, and KorQuAD model which utilizes it.\\
        \item [-] model/Q\&A folder : This is a a folder in which the model we created using our own timeline data is stored to pre-train the final model.\\
        \item [-] model/bert folder : This is a folder that contains a Korean exclusive BERT model that used Google's sentencepiece, Wikipedia, and news data to learn 180 million sentences of vocabulary (subwords).\\
        \item [-] model/elmo folder : This is a folder that contains Korean Wikipedia, Naver movie review corpus, and models learned with KorQuAD.\\
        
\end{enumerate}

\section{Use Cases}
\subsection{Loading}
\begin{figure}[h]
\centerline{\includegraphics[width=4cm, height=7cm]{loading.png}}
\caption{Loading page}
\label{fig}
\end{figure}
Figure 4: Loading Page is a page that a user will see after turning on the application.This page is turned on only while the application loads the elements needed to operate, and automatically moves on to the next page at the end of loading.\\

\clearpage
\subsection{Role}
\begin{figure}[h]
\centerline{\includegraphics[width=5cm, height=8cm]{RolePage1.png}}
\caption{Role Page 1}
\label{fig}
\end{figure}
\\
Figure 5: Role Page 1 shows the record function of MEMENTO, acknowledging user that making a recording will help user prevent dementia.

\begin{figure}[h]
\centerline{\includegraphics[width=5cm, height=8cm]{RolePage2.png}}
\caption{Role Page 2}
\label{fig}
\end{figure}
\\
Figure 6: Role Page 2 shows the quiz function of MEMENTO, acknowledging user that they can play quizzes with the AI speaker NUGU based on the activities, and recordings they have made.
\\\\\\\\
\begin{figure}[h]
\centerline{\includegraphics[width=5cm, height=8cm]{RolePage3.png}}
\caption{Role Page 3}
\label{fig}
\end{figure}
\\
\indent Figure 7: Role Page 3 shows the LG appliance manage function of MEMENTO, acknowledging user that they can register their home appliances and even control them through the application.

\begin{figure}[h]
\centerline{\includegraphics[width=5cm, height=8cm]{RolePage4.png}}
\caption{Role Page 4}
\label{fig}
\end{figure}
\\Figure 8: Role Page 4 shows the dementia prevention game page function of MEMENTO, acknowledging user that they can activate their brain by playing games through the application.

\clearpage
\begin{figure}[h]
\centerline{\includegraphics[width=5cm, height=8cm]{RolePage5.png}}
\caption{Role Page 5}
\label{fig}
\end{figure}
\\
Figure 9: Role Page 5 shows the useful information about Youngzheimer, acknowledging users that they can obtain information about dementia which might not be a familiar topic to younger generations.
\\
\subsection{Login}
\begin{figure}[h]
\centerline{\includegraphics[width=5cm, height=8cm]{Login.png}}
\caption{Login Page}
\label{fig}
\end{figure}
\\
Figure 10: Login Page is a page which allows users to log in by entering their ID and password.
\\\\\\
\begin{figure}[h]
\centerline{\includegraphics[width=5cm, height=8cm]{LoginSuccess.png}}
\caption{Login Success Pop-up}
\label{fig}
\end{figure}
\\
\indent Figure 11: Login Success Pop-up is a pop-up that acknowledges user that their login attempt at Figure 10: Login Page is successful.
\\
\subsection{Home}
\begin{figure}[h]
\centerline{\includegraphics[width=5cm, height=8cm]{CalendarPage.png}}
\caption{Calendar page}
\label{fig}
\end{figure}
Figure 12: Calendar Page serves as a home page of MEMENTO. Users can freely choose the year, month, and day. Touching the '영츠하이머 그게 뭔데?' button leads the user to Figure 35: Youngzheimer, and touching the '영츠하이머 자가 진단!' button leads the user to Figure 39: Self-Diagnosis Page 1 and touching the '영츠하이머 예방 게임!' button leads the user to Figure 29: Youngzheimer Prevention Game. 

\subsection{Recording}
\begin{figure}[h]
\centerline{\includegraphics[width=4cm, height=7cm]{Record.png}}
\caption{Record page}
\label{fig}
\end{figure}
Figure 13: Record Page is a page where a user can make their own recording based on their daily activities, emotions, and events. As soon as a user touches the record button, the recording begins. After touching the button, user can freely record their daily activity, emotion, and events. By touching the button again will finish the recording, and the application will automatically send the recorded file, the location information of the user using GPS function, and the time information to the server. 

\begin{figure}[h]
\centerline{\includegraphics[width=4cm, height=7cm]{RecordCoffee.png}}
\caption{Recording 1}
\label{fig}
\end{figure}
Figure 14: Recording 1 is a page where it shows the current recorded information in real-time.
\\\\\\\\\\
\begin{figure}[h]
\centerline{\includegraphics[width=4cm, height=7cm]{RecordCoffeeSuccess.png}}
\caption{Recording 1 Successful}
\label{fig}
\end{figure}
\\
\indent Figure 15: Recording 1 Successful is a pop-up that a user will see after finishing recording. This pop-up acknowledges user that a recording has been successfully saved and it has been applied to the Figure 26: Timeline.

\begin{figure}[h]
\centerline{\includegraphics[width=4cm, height=7cm]{RecordLibrary.png}}
\caption{Recording 2}
\label{fig}
\end{figure}
Figure 16: Recording 2 is a page where it shows the current recorded information in real-time. After finishing recording, it will be automatically applied to the Figure 26: Timeline.

\clearpage
\begin{figure}[h]
\centerline{\includegraphics[width=5cm, height=8cm]{RecordLibrarySuccess.png}}
\caption{Recording 2 Successful}
\label{fig}
\end{figure}
Figure 17: Recording 2 Successful is a pop-up that a user will see after finishing recording. This pop-up acknowledges user that a recording has been successfully saved and it has been applied to the Figure 26: Timeline.
\\
\begin{figure}[h]
\centerline{\includegraphics[width=5cm, height=8cm]{RecordFail.png}}
\caption{Recording Fail}
\label{fig}
\end{figure}
\\
\indent Figure 18: Recording Fail is a pop-up that a user will see when the application has failed to finish recording. This pop-up acknowledges user that a recording has not been successfully saved and it has not been applied to the Figure 26: Timeline.

\subsection{LG}
\begin{figure}[h]
\centerline{\includegraphics[width=5cm, height=8cm]{LGApplianceBlank.png}}
\caption{LG Home Appliance Page}
\label{fig}
\end{figure}
\\
Figure 19: LG Home Appliance Page is a page where a user can manage their own LG home appliances. This page is originally blank, and user can later add their LG home appliances by pressing the + button located on the bottom right of the page. 
\begin{figure}[h]
\centerline{\includegraphics[width=5cm, height=8cm]{LGAddAppliance.png}}
\caption{Adding LG Home Appliance Page}
\label{fig}
\end{figure}
\\
\indent Figure 20: Adding LG Home Appliance Page is a page where user can enter the serial number of the LG appliance, choose the type of the appliance, and choose what time and what day will the LG home appliances are planned to be used.
\\
\clearpage
\begin{figure}[h]
\centerline{\includegraphics[width=5cm, height=7.5cm]{LGAddAppliance2.png}}
\caption{Adding LG Home Appliance Successful}
\label{fig}
\end{figure}
\\
Figure 21: Adding LG Home Appliance Successful is a pop-up that appears after a user typed all the requirements for registering LG home appliances.
\\\\\\
\begin{figure}[h]
\centerline{\includegraphics[width=5cm, height=7.5cm]{LGAddApplianceError.png}}
\caption{Adding LG Home Appliance Unsuccessful}
\label{fig}
\end{figure}
\\
\indent Figure 22: Adding LG Home Appliance Unsuccessful is a pop-up that appears when a user attempts to register LG home appliance without entering all the information required. To successfully register and appliance, user must go back to Figure 20: Adding LG Home Appliance Page and finish entering all the information.
\\\\
\begin{figure}[h]
\centerline{\includegraphics[width=5cm, height=7.5cm]{LGAppliance1.png}}
\caption{LG Home Appliance Page(Appliance Registered 1)}
\label{fig}
\end{figure}
\\
\indent Figure 23: LG Home Appliance Page(Appliance Registered 1) is a page that will appear after user has successfully registered their own LG home appliances in Figure 20: Adding LG Home Appliance Page. Users can easily check what type of appliances they have registered, what time and day these appliances are scheduled to be operated.
\\
\begin{figure}[h]
\centerline{\includegraphics[width=5cm, height=7.5cm]{LGAppliance2.png}}
\caption{LG Home Appliance Page(Appliance Registered 2)}
\label{fig}
\end{figure}
\\
Figure 24: LG Home Appliance Page(Appliance Registered 2) is a page that will appear after user has successfully registered more than 1 LG home appliances in Figure 20: Adding LG Home Appliance Page. Users can easily check what type of appliances they have registered, what time and day these appliances are scheduled to be operated.
\\
\subsection{Timeline}
\begin{figure}[h]
\centerline{\includegraphics[width=4.5cm, height=7.5cm]{TimelineGenerated.png}}
\caption{Calender - Timeline Generated}
\label{fig}
\end{figure}
Figure 25: Calendar - Timeline Generated is a page that a user will see after successfully creating a record from Figure 13: Record or registered a LG home appliance from Figure 20: Adding LG Home Appliance Page Touching the date leads the user to Figure 26: Timeline.
\\
\begin{figure}[h]
\centerline{\includegraphics[width=4.5cm, height=7.5cm]{Timeline.png}}
\caption{Timeline}
\label{fig}
\end{figure}
\\
Figure 26: Timeline is a page that a user will see after successfully registering a LG home appliance from Figure 20: Adding LG Home Appliance Page.
\\\\
\begin{figure}[h]
\centerline{\includegraphics[width=4.5cm, height=7.5cm]{TimelineLGActivate.png}}
\caption{Activate LG Home Appliance}
\label{fig}
\end{figure}
\\
Figure 27: Activate LG Home Appliance is a pop-up that appears when a user touches a timeline of their LG home appliances. Pop-up will acknowledge user that the application will activate the appliance scheduled from settings made in the Figure 20: Adding LG Home Appliance Page.
\\
\begin{figure}[h]
\centerline{\includegraphics[width=4.5cm, height=7.5cm]{TimelineQuizGenerate.png}}
\caption{Generate Quiz}
\label{fig}
\end{figure}
\\
Figure 28: Generate Quiz is a pop-up that appears when a user touches a '퀴즈 생성' button located in the bottom middle of the Figure 26: Timeline page. After generating quiz, user ready to play quiz with NUGI ai speaker.
\\
\clearpage
\subsection{Youngzheimer Prevention Game}
\begin{figure}[h]
\centerline{\includegraphics[width=4cm, height=7cm]{PreventionGame.png}}
\caption{Youngzheimer Prevention Game}
\label{fig}
\end{figure}
\\
Figure 29: Youngzheimer Prevention Game is a page where a user can conduct three types of games, which are games were jointly studied by the Catholic University of Korea's St. Mary's Hospital Brain Health Center and AriaCare, that can improve concentration, memory, and calculation ability.\\

\begin{figure}[h]
\centerline{\includegraphics[width=4cm, height=7cm]{PreventionGame1.png}}
\caption{Concentration Game}
\label{fig}
\end{figure}
Figure 30: Concentration Game is where a user can improve their concentration. A user will be given a question which a word meaning random color is printed, but the word is painted in a different color than the color the word refers to. The user must shout the color on which the actual letter is painted to be recognized as the correct answer. If a user answers a correct answer, the application will show Figure 34: Correct, and will show Figure 35: Wrong if the user is incorrect.
\\
\begin{figure}[h]
\centerline{\includegraphics[width=4cm, height=7cm]{PreventionGame2.png}}
\caption{Memory Game}
\label{fig}
\end{figure}
\\
Figure 31: Memory Game is where a user can improve their memory. A user has to choose a correct category after looking at a word. For example, if a "TV" is given, and user has to choose the correct category of "TV" in Figure 32: Memory Game Answer Selection.
\\
\begin{figure}[h]
\centerline{\includegraphics[width=4cm, height=7cm]{PreventionGame2A.png}}
\caption{Memory Game Answer Selection}
\label{fig}
\end{figure}
\\
Figure 32: Memory Game Answer Selection is where a user has to choose correct category of a word they have seen in Figure 31: Memory Game. If a user answers a correct answer, the application will show Figure 34: Correct, and will show Figure 35: Wrong if the user is incorrect.
\\
\clearpage
\begin{figure}[h]
\centerline{\includegraphics[width=5cm, height=8cm]{PreventionGame3.png}}
\caption{Calculation Game}
\label{fig}
\end{figure}
\\
Figure 33: Calculation Game is where a user can improve their calculation ability. A User have to solve simple four-point arithmetic problems in mind, and must choose the correct answer. If a user answers a correct answer, the application will show Figure 34: Correct, and will show Figure 35: Wrong if the user is incorrect.
\\
\begin{figure}[h]
\centerline{\includegraphics[width=5cm, height=8cm]{Correct.png}}
\caption{Correct}
\label{fig}
\end{figure}
\\
Figure 34: Correct is a pop-up what a user will see when they choose the correct answer from either Figure 30: Concentration Game, Figure 31: Memory Game, Figure 33: Calculation Game.
\\\\
\begin{figure}[h]
\centerline{\includegraphics[width=5cm, height=8cm]{Wrong.png}}
\caption{Wrong}
\label{fig}
\end{figure}
\\
Figure 35: Wrong is a pop-up what a user will see when they choose the incorrect answer from either Figure 20: Concentration Game, Figure 21: Memory Game, Figure 22: Calculation Game.
\subsection{Youngzheimer}
\begin{figure}[h]
\centerline{\includegraphics[width=5cm, height=8cm]{Youngzheimer.png}}
\caption{What is Youngzheimer?}
\label{fig}
\end{figure}
\\
Figure 36: What is Youngzheimer? is a page where a user can see useful information about dementia, such as current state of dementia, seriousness of dementia, and dementia prevention methods.

\clearpage
\begin{figure}[h]
\centerline{\includegraphics[width=5cm, height=8cm]{YoungzheimerInfo1.png}}
\caption{Current State of Dementia}
\label{fig}
\end{figure}
Figure 37: Current State of Dementia is a page where a user can see the overall status of dementia such as total number of worldwide patients of dementia, proportion of dementia by it's type, and others.
\\
\begin{figure}[h]
\centerline{\includegraphics[width=5cm, height=8cm]{YoungzheimerInfo2.png}}
\caption{Dangers of Dementia}
\label{fig}
\end{figure}
\\
Figure 38: Dangers of Dementia is a page where a user can see the seriousness of dementia.
\\\\\\\\\\\\
\begin{figure}[h]
\centerline{\includegraphics[width=5cm, height=8cm]{YoungzheimerInfo3.png}}
\caption{Dementia Prevention Method}
\label{fig}
\end{figure}
\\
Figure 39: Dementia Prevention Method is a page where a user can obtain useful information such as dementia support facility location, support system, how to make contacts, foods and exercise good for dementia, and others.

\subsection{Self-Diagnosis}
\begin{figure}[h]
\centerline{\includegraphics[width=5cm, height=8cm]{DementiaTest1.png}}
\caption{Self-Diagnosis Page 1}
\label{fig}
\end{figure}
Figure 40: Self Diagnosis Page 1 is a page where user can conduct 14 questions created by Central Dementia Center of the Ministry of Health and Welfare. User can Scroll down to keep on to next question, leading user to Figure 41: Self Diagnosis Page 2.

\begin{figure}[h]
\centerline{\includegraphics[width=5cm, height=8cm]{DementiaTest2.png}}
\caption{Self-Diagnosis Page 2}
\label{fig}
\end{figure}
Figure 41: Self Diagnosis Page 2 is a page where user can continue to conduct 14 questions from Figure 40: Self Diagnosis Page 1. Users that checked less than six of the questions will be lead to Figure 42: Safe, and users that checked 6 or more questions are lead to Figure 43: Danger.

\begin{figure}[h]
\centerline{\includegraphics[width=5cm, height=8cm]{Safe.png}}
\caption{Safe}
\label{fig}
\end{figure}
Figure 42: Safe is a pop-up that appears when user have finished Self Diagnosis in Figure 40: Self Diagnosis Page 1 and Figure 41: Self Diagnosis Page 2 with less than six checks. Users are considered safe from dementia with this result, and are recommended to stay this way.
\\
\begin{figure}[h]
\centerline{\includegraphics[width=5cm, height=8cm]{Danger.png}}
\caption{Danger}
\label{fig}
\end{figure}
\\
Figure 43: Danger is a pop-up that appears when user have finished Self Diagnosis in Figure 40: Self Diagnosis Page 1 and Figure 41: Self Diagnosis Page 2 with six or more checks. Users are considered dangerous and are strongly recommended to perform medical diagnosis for more accurate results. 

\section{Conclusion \& Discussion}
\indent In documentation part, there were some problems with Overleaf, LaTeX. IEEE Conference Template was somehow making a glitch in the blank spaces between tables, figures, and text areas. I have made a ton of research and asked for help to my colleagues and even professor Won, but none of them found an answer to this problem. I tried my best in my ability, and so far the results are quite satisfying. If I knew better about LaTeX or had any experiences before this semester, the outcome would have looked better. Overall, I think it was a great opportunity to experience a brand new program in documentation.\\
\indent For the Frontend part, it was difficult to connect React Native and xcode. This is because the process of building code on our iPhone device did not go smoothly. There were many errors in the process of building the JavaScript code from a mobile phone with xcode, and it took a lot of time to solve it.\\
\indent For the Backend part, the most difficult thing was the connection between ai and server. MEMENTO's server is composed of java-based springs, and ai is composed of Python, so it was not easy to connect the two. After trying various methods, we decided to build a flask server that executes the ai model. To be honest, the establishment of the flask server was easy, but the process of distributing the program to the AWS EC2 instance was very difficult. We needed to run a 1.15GB size model, but our EC2 instance didn't have enough memory, so it kept getting killed. To solve this problem, swap memory was implemented to overcome the limitations of physical memory capacity.\\
\indent Although there were many barriers building MEMENTO, but our team has found a way. Everyone seems to be satisfied by the result and we are actually planning to move on to the next step. We are currently planning to add more features such as providing analysis function by extracting keywords from the recordings a user have made, and later provide weekly, monthly, and yearly keyword analysis to the user. When we are done adding more features and better stabilize the app, we are planning to register patent to the Korean Intellectual Property Office, the Patent and Trademark Office, the Patent Office. Once this is done, we are going to register an Apple Developer ID and launch our app to Apple Store. 
\\
\begin{thebibliography}{00}
\bibitem{b1} The brain in your pocket: Evidence that Smartphones
are used to supplant thinking. Nathaniel Barr, Gordon Pennycook, Jennifer A. Stolz, Jonathan A. Fugelsang, CA: Department of Psychology, University of Waterloo, Canada, 2015.
\bibitem{b2} WHO - Global action plan on the public health response to dementia 2017 - 2025 
\bibitem{b3} Central Dementia Center \& National Medical Center \& Ministry of Health and Welfare - Korean Dementia Observatory 2019 - 2021
\bibitem{b4} Is it hard because of diet failure, insomnia, and anger? \\Writing a "Healthy Diary" might help you out :\\ https://www.joongang.co.kr/article/25039335\#home
\bibitem{b5} SILVIA - Everything about brain health care : https://silvia.io/
\bibitem{b6} Dementia care application "치매체크" by Ministry of Health and\\ Welfare : https://play.google.com/store/apps/details?id=kr.co.inergy.\\selftest\&hl=ko\&gl=US
\bibitem{b7} Neuronation - A fitness for your brain : https://www.neuronation.com/
\bibitem{b8} Young Onset Dementia Clinic - https://www.snubh.org/dh/main/index.do?\\DP\_CD=DCD21\&MENU\_ID=003004
\bibitem{b9} Dementia Prevention Smartphone games :\\https://www.100ssd.co.kr/news/articleView.html?idxno=68732
\bibitem{b10} 99 stories of dementia - Korean Society of Dementia :\\https://www.dementia.or.kr/general/bbs/index.php?code=story\&category\\=\&gubun=\&page=6\&number=1060\&mode=view\&keyfield=\&key=
\bibitem{b11} The Lancet Commisions : Dementia prevention, intervention, and care: 2020 report of the Lancet Commission
\bibitem{b12} Self Dementia Test - National Health Insurance Service\&Ilsan Hospital:\\https://www.nhimc.or.kr/health/discomfort/health\_online\_dimentia.do
\bibitem{b13} GERONTOLOGICAL - Personal Journal Keeping and Linguistic Complexity Predict Late-Life Dementia Risk:\\ The Cache County Journal - Pilot Study :\\ https://academic.oup.com/psychsocgerontology/article/72/6/991/2632021
\bibitem{b14} Young-onset dementia - Alzheimer's Society :\\https://www.alzheimers.org.uk/about-dementia/types-dementia/young-onset-dementia
\bibitem{b15} Dementia - Early Signs : https://www.betterhealth.vic.gov.au/health/\\conditionsandtreatments/dementia-early-signs
\bibitem{b16} Dementia Risk Factors - Stanford Medicine :\\https://stanfordhealthcare.org/medical-conditions/brain-and-nerves/dementia/risk-factors.html
\bibitem{b17} Can You Die From Dementia? - healthline :\\ https://www.healthline.com/health/dementia/can-you-die-from-dementia
\bibitem{b18} Early Care Syndrome - \\http://www.sellernews.co.kr/news/articleView.html?idxno=130
\bibitem{b19} Trend Keyword 'Healthy Pleasure' \& 'Early Care Syndrome' - https://cosinkorea.com/mobile/article.html?no=42875
\end{thebibliography}

\tableofcontents

\listoffigures

\listoftables

\end{document}